\chapter{Quality Assurance Manual}
\section{Introduction}
In recent years, quality assurance and control (QA/QC) has become 
fundamental to the production of analytical data for scientific 
research. The purpose of this manual is to describe the quality 
assurance program employed at the University of Nebraska Water Sciences 
Laboratory (WSL). The Water Sciences Laboratory strives to apply 
appropriate elements of this program to all research and analytical 
activities. This manual provides Laboratory personnel and other 
interested parties with a description of our policies for maintaining 
analytical quality assurance. Quality Assurance Project Plans (QAPP) 
define projectspecific QA policies for research and service conducted 
through the Water Sciences Laboratory.



\section{Facility}
The WSL is an analytical facility within the Nebraska Water Center and the Daugherty Water for Food Global Institute. 
It was originally funded through Institute of Agriculture and Natural Resources (IANR). In 1990, IANR initiated a \$400,000 renovation of an existing East Campus building to provide a laboratory for specialized analyses 
including trace levels of agrichemical compounds and environmental 
isotopes. The 6,000 square foot facility consists of six laboratories, 
several offices, a conference room, and staff/graduate student areas. 

The location and working environment promote collaborative research on 
water-related projects involving professors and students from Agronomy, 
Biological Sciences, Geology, Chemistry, Biological Systems Engineering, 
Entomology, Civil Engineering, and the School of Natural Resource 
Sciences and the Conservation and Survey Division. The Laboratory also 
serves as the State's clearinghouse for all statewide ground water 
pesticide and nitrate data, and is closely associated with the Nebraska 
Department of Environmental Quality, State Health Department, Nebraska 
Natural Resources Commission, and local Natural Resource Districts.



\section{Objectives and Quality Policies }
The objective of the WSL Quality Assurance Program is to foster 
accurate, precise, and reliable analytical results for all procedures 
performed at this facility. Implementation of this program includes 
managerial, statistical, investigative, preventative and corrective 
techniques to maximize data quality at the minimum additional cost. The 
program must be cost effective, and at the same time enable the WSL to 
meet or exceed both project and non-project data quality standards. 
Specific objectives to promote quality assurance include: 

\newcounter{numberedCntF}
\begin{enumerate}
	\item development and utilization of rugged and proven analytical methods adapted from published and standardized procedures 
	\item training of appropriate laboratory personnel in basic QA/QC measures and laboratory-specific methods 
	\item establishment of performance standards to compare with routine data quality
\setcounter{numberedCntF}{\theenumi}
\end{enumerate}


\begin{enumerate}
\setcounter{enumi}{\thenumberedCntF}
	\item establishment of procedures for method/procedure modification to improve data quality
	\item monitoring and documenting routine analytical performance
	\item participation in appropriate performance evaluation programs
	\item establishment of performance standards for laboratory personnel 
\setcounter{numberedCntF}{\theenumi}
\end{enumerate}

\noindent
In general, WSL policies emphasize the prevention of problems rather 
than detection and correction of problems after they occur. The WSL 
shall use published standardized methods and provide written procedures, 
including basic QA/QC requirements, to staff for all routine methods and 
activities influencing data quality. New methods will be verified using 
suitable test samples or reference materials, and compared to previously 
validated methods if possible. The WSL shall retain copies of all 
supporting documentation, including analytical results, for a time 
period specified by each project. If necessary, results are held 
indefinitely to verify the actions taken for each sample analyzed at the 
facility. A comprehensive calibration and maintenance program optimizes 
instrument performance and data quality. Reagents and supplies used 
shall be of appropriate grade for the procedure, and gravimetric and 
volumetric apparatus shall be of a suitable class and calibrated as 
necessary. 

Analytical data quality objectives (DQO) for environmental research 
projects will define the confidence level required, and determine the 
level of reliability, precision, accuracy, detection limits, and 
validation methods. Although the level of reliability, precision, and 
accuracy required for most analyses varies according to the method and 
analyte, data quality is to be kept as high as practical on a day-to-day 
basis. 



\section{Sample Collection, Handling, and Login}
Sampling is often performed by field-trained Water Sciences 
Laboratory personnel following a sampling plan defining the objectives 
or purposes for sample collection and analysis. At present, the WSL does not emply a field collection specialist so samples are collected by the submitting agency. The written plan 
provides QA documentation, describes guidelines, step-by-step sampling 
instructions, references to standard operating procedures (SOPs), and 
ensures that sampling is accomplished as planned. The sampling plan 
addresses the matrix to be sampled, collection method, statistical 
requirements, containers, preservation methods, and handling procedures. 
Specific instructions for identification of samples, as well as other 
information to be included on the containers, as well as description, 
labeling and frequency of field quality control (QC) samples are also 
included in the plan. Sample handling and transport may also be 
referenced in an SOP. If chain-of-custody forms are used to document 
sample collection and transport, an example form is included in the 
plan. Finally, any special treatment, holding and disposal requirements, 
and routing of results is also addressed in the sampling plan.

WSL staff accepting samples and field records are responsible for 
initiating the laboratory custody record and insuring that the handling 
and condition of each sample is documented. Samples suspected of being 
inferior will be noted and, at the discretion of the analyst or 
Laboratory Director, may be rejected for analysis. The sample collector 
is contacted to request a replacement sample. If an additional sample is 
unavailable, questionable samples may be analyzed but the results will 
be flagged and appropriate narrative provided.

Sample receipt and login includes assignment of a unique laboratory 
identification number. Sample receipt is currently recorded on hard-copy 
tracking forms, and all sample information is then entered onto a 
computer-based Laboratory Information Management System (WSLims). The 
WSLims computer program has been created in-house using Borland C++ 
(v.4.5) programming language, Object-Windows (v.2.5) graphical-user 
interface (GUI), and Borland Database Engine (v.2.0) for the database 
structure and code. WSLims assigns the unique laboratory ID number used 
to identify and track samples throughout processing and analysis. Field 
samples are batched in groups no larger than twenty with four laboratory 
quality control (QC) samples added to each batch.

 

\section{Analytical Procedures}
All routine methods at the Water Sciences Laboratory are in the form of 
numbered standard operating procedures (SOPs). The written format for 
standard operating procedures is described in \hyperlink{Analytical}{Part II} of 
the Water Sciences Laboratory Procedures and Analytical Methods manual. 
The format includes sections for method references, scope, basic 
principles, apparatus, safety, step-by-step procedures, calculations, 
statistics, quality assurance, and additional information helpful for 
utilizing the procedure. The list of SOPs continues to grow, and 
procedures are updated as needed to incorporate changes and improvements 
in our analytical methodology. In general, routine analytical methods 
must meet realistic objectives with respect to sensitivity, accuracy, 
reliability, precision, interferences, matrix effects, limitations, 
costs, and the time required. 

Most of the analytical procedures used are based on published and 
standardized methods found in: \textit{Standard Methods for the 
Examination of Water and Wastewater }(APHA ,1998); \textit{Methods for 
the determination of organic compounds in drinking water }(USEPA, 
1988;1992); \textit{Test Methods for Evaluating Solid Waste}, 
\textit{Physical/Chemical Methods, SW-846} (USEPA, 1986, and current 
updates)\textit{ Methods for chemical analysis of water and waste} 
(USEPA, 1983), \textit{ASTM Annual Book of Standards }(ASTM, 1991), 
\textit{Techniques of Water-Resources Investigations }(USGS, 1989), 
and \textit{Methods of Soil Analysis (}ASA, 1986). When standardized 
methods are not available or are unsuitable, in-house methods are 
developed and often based on procedures found in scientific publications 
such as \textit{Analytical Chemistry, Journal of the Association of 
Analytical Chemists, Journal of Chromatography,} and a wide variety of 
other peer-reviewed publications. Routine methods are validated using 
test samples and standards, compared to previously-used methods if 
applicable, and if found acceptable are described in a written SOP. 

\section{Instrument Calibration and Maintenance}
Instrumentation housed at the Water Sciences Laboratory include two 
light gas isotope ratio mass spectrometers and three high-vacuum 
preparation systems which are used for highly precise measurements of 
the variations in the amounts of the stable isotopes of nitrogen in 
nitrate, as well as the stable isotopes of hydrogen and oxygen in water 
for tracing water movement in hydrologic systems. Two gas 
chromatograph/mass spectrometer (GC/MS) quadrupole systems are used for 
measuring trace levels of pesticides and degradation products of 
pesticides, gasoline oxygenates, organic acid derivatives, algal 
metabolites and other volatile thermally-labile compounds. Other gas 
chromatographs, two ion chromatographs, and an HPLC system are used for 
measuring dissolved gases, dissolved ions, and polar organic compounds 
in ground and surface water. A liquid chromatograph (LC) interfaced with 
an ion trap tandem mass spectrometer (LC/MS/MS) provides the capability 
to determine explosives residues and RDX degradation products (MNX and 
TNX), acetamide degradation products, pharmaceutical compounds, and 
other polar organics that are not suitable for determination by GC/MS. A 
GC with a micro-electron capture detector interfaced with a vacuum 
extraction system is used in ultra-trace level determination of 
chlorofluorocarbons (CFCs) for ground water age-dating. An inductively 
coupled plasma mass spectrometry (ICP-MS) is used to determine water 
hardness and other metals, including isotope analysis. Other analytical 
equipment includes a supercritical fluid extractor (SFE), carbon 
analyzers, and spectrophotometers, as well as some older radiochemical 
instrumentation used for naturally occurring isotopes. All of these 
analyses are under the management of this QA Manual. A list of 
applicable Standard Operating Procedures (SOPs) for current methods is 
provided as Appendix III.

Calibration frequency is a function of the instrument and the 
procedures, and the SOPs specify the minimum required. Calibration is 
required before running samples on any equipment with frequent (?5\%) 
calibration checks. Complete recalibration is recommended at the 
beginning of every run analyzing a maximum 2 complete batches, with more 
frequent recalibrations necessary if increased variability is observed. 
Results with the highest level of certainty require complete calibration 
before and after the analysis, and new calibration curves must be 
checked against previous curves to determine if the instrument and 
standards are giving acceptable and similar responses. Where possible, 
the calibration is checked against an independently prepared secondary 
or reference standard for additional verification. See each specific SOP 
for further clarification.

Scheduled preventative maintenance varies according to the instrument 
and the procedures used, and serves to optimize performance while 
reducing problems. Most instruments in the Water Sciences Laboratory are 
housed in humidity and temperature-controlled rooms where exposure to 
corrosive fumes, dust, and vibrations is minimized, and maintenance 
costs are likewise minimized. Preventative maintenance is often outlined 
in the instrument manuals, and it is recommended that this list be 
transferred to the instrument notebook together with a recommended 
schedule for reference. Routinely replaced components should be listed 
in the notebook and kept in stock to minimize downtime. The Water 
Sciences Laboratory does not have any maintenance contracts in place, 
and does the primary maintenance and servicing of the instruments. 

Troubleshooting and repair procedures are performed when an instrument 
malfunctions. Diagnostic procedures are usually found in the instrument 
manual, notebook, or may be obtained from the instrument manufacturer. 
All repairs and maintenance are performed by trained and qualified 
personnel from the instrument manufacturer, university instrument shops, 
or the Water Sciences Laboratory. 

Calibration runs and instrumental maintenance are documented in 
the same bound instrument notebooks. The instrument notebook should 
contain all pertinent instrument identification information on the first 
page, including manufacturer, model and serial numbers, UNLID numbers, 
installation date, warranty information, room and building numbers, and 
any other relevant information. Calibration record entries include the 
date and time, the sample batch, instrument identification and location, 
calibration procedure used, the instrument operator and the results of 
the calibration. All maintenance work, whether preventative or 
unscheduled, is documented in the instrument notebook. Maintenance 
record entries include the date and time, symptoms, maintenance or 
repair details, date repair completed, parts replaced, name or initials 
of person who performed the work, and any other relevant information. 
The current instrument notebook is to remain stationed with the 
appropriate instrument for continuous reference and updating. 



\section{Laboratory Quality Control Checks}
Quality control (QC) includes all procedures followed to ensure 
that the accuracy of the data generated are known to a stated degree of 
probability. QC encompasses instrument calibration, personnel training, 
and use of pure reagents and certified standards. QC checks (samples) 
are used to monitor the performance of the analytical system. All QC 
samples, whether laboratory or field, are logged into the WSLims 
database and assigned a unique laboratory ID\#. Thus, during processing 
and analysis QC samples are indistinguishable from other samples. Checks 
for laboratory quality control include the following in all routine 
standard analyses:


\begin{table}[]
	\centering
	\caption{Laboratory quality controls}
	\begin{tabular}{@{}lcc@{}}
		\toprule
		Description                 & Abbreviation & Frequency    \\ \midrule
		Laboratory Reagent Blank    & LRB          & at least 5\% \\
		Laboratory Fortified Blank  & LFB          & at least 5\% \\
		Laboratory Duplicate        & LD           & at least 5\% \\
		Laboratory Fortified Matrix & LFM          & up to 5\%    \\ \bottomrule
	\end{tabular}
\end{table}

\noindent
For trace-level analysis the following additional checks may be added:


\begin{table}[]
	\centering
	\caption{Additional checks}
	\begin{tabular}{@{}ll@{}}
		\toprule
		Description                  & \multicolumn{1}{c}{Frequency}    \\ \midrule
		Isotope/Internal standards   & \multicolumn{1}{c}{every sample} \\
		Surrogates                   & every sample                     \\
		Reference/Certified standard & \multicolumn{1}{c}{as available} \\
		Instrument replicates        & \multicolumn{1}{c}{at least 5\%} \\
		Batch replicates             & at least 5\%                     \\
		Solvent replicates           & at least 5\%                     \\
		Spike check                  & at least 5\%                     \\
		Performance evaluation       & as available                     \\ \bottomrule
	\end{tabular}
\end{table}

\noindent
Depending on the project, the Water Sciences Laboratory also analyzes 
and evaluates field QC samples including:

\begin{table}[]
	\centering
	\caption{My caption}
	\begin{tabular}{@{}lcc@{}}
		\toprule
		Description                    & Abbreviation & Frequency    \\ \midrule
		Field Duplicate samples        & FD1          & at least 5\% \\
		Field Reagent Blanks           & FRB          & at least 5\% \\
		External laboratory duplicates & FDX          & up to 5\%    \\
		Field equipment blanks         & FEQ          & up to 5\%    \\ \bottomrule
	\end{tabular}
\end{table}


Most analyses involve the generation of multilevel or multi-standard 
calibration curves immediately prior to sample analysis. The number of 
calibration levels range from two to six-point, depending on the 
protocol, with a higher number of levels used in more critical 
trace-level analytical work. Samples with analyte concentrations above 
the calibration curve are normally rerun after adjusting either the 
sample concentration or the calibration range to produce a response 
falling within the calibration range. Calibrations are often checked 
using an externally prepared reference sample or certified standard.

Analytical precision and accuracy are monitored through the use of 
Shewhart statistical parameters (I, R, and P) and quality control 
charts. Control charts are usually generated to visually monitor 
duplicate ranges (R), spike recovery (P), and matrix-spike recovery (P).

 
% Duplicate range
\begin{equation}
R = \abs{FD1-FD2}
\end{equation}

% I 
\begin{equation}
I = \dfrac{\abs{FD1 - FD2}}{FD1 + FD2}
\end{equation}

\noindent
Upper control limits (UCL) for the range (R) of duplicate analyses is determined by:

% UDL 
\begin{equation}
Upper Control Limit (UCL) = D_{4}\dfrac{\sum_{i=1}^{n} R_{i}}{n}
\end{equation}

\noindent
"R" values for duplicate analyses are generally calculated, tabulated, 
and graphed using WSLims or spreadsheet software (Excel, Microsoft 
Corporation).

\noindent
Accuracy is monitored using percent recovery (P) in fortified blanks 
(LFB) and matrix spike (LFM) samples, and may be checked using standard 
reference materials (SRM) and performance evaluation (PE) samples. 

% Percent Recovery
\begin{equation}
P_{LFB} = 100 (\dfrac{measured}{known})
\end{equation}

\begin{equation}
P_{LFM} = 100 (\dfrac{measured-background}{spike})
\end{equation}

\noindent
Upper and lower control limits for recovery are determined by:

% Control Limits
\begin{equation}
UCL = \frac{\sum_{i=1}^{n} P_{i}}{n} + 3\sqrt{\frac{\sum \left (P_{i} - \frac{\sum_{i=1}^{n} P_{i}}{n} \right )^2}{n-1}}
\end{equation}

\begin{equation}
LCL = \frac{\sum_{i=1}^{n} P_{i}}{n} - 3\sqrt{\frac{\sum \left (P_{i} - \frac{\sum_{i=1}^{n} P_{i}}{n} \right )^2}{n-1}}
\end{equation}

Qualitative identification and confirmation of contaminants, or absence 
thereof, is done by comparison of the results with those of a known 
amount of standard reference material or by comparison to a second, 
well-characterized method. For assay and impurity tests, specificity is 
demonstrated by the resolution of the two closest eluting compounds. If 
impurities are available, it must be demonstrated that the assay is 
unaffected by the presence of spiked materials (impurities and/or 
excipients). If impurities are not available, the test results are 
compared to a second well-characterized procedure. This is further 
described in the specific analyte SOP. 

A method detection limit (MDL) is defined as the minimum concentration 
that can be measured with a 99\% confidence that the concentration is 
greater than zero. MDLs are determined for all routine analytical 
methods from analysis of a prepared test sample in a matrix similar to 
typical unknown samples. The procedure used is taken directly from EPA 
Federal Register (1989) Pt.136 Appendix B, Definition and procedure for 
the determination of the method detection limit - Rev. 1.11. All new or 
revised methods are subjected to MDL tests before use on unknown 
samples. MDLs for trace-level analyses are repeated annually, or more 
frequently if necessary, to confirm sensitivity. Reporting limits, or 
Quantitation Limits (QL), are typically set at 3 to 5 times the 
concentrations obtained from method detection limit tests to compensate 
for additional uncertainty when handling unknown samples. 

Microbiology samples are generally not processed by the Water Sciences 
Laboratory, and thus no specific parameters are in place for such 
samples. 



\section{Nonconformity and Corrective Action}
QC nonconformity may indicate an analytical problem requiring corrective 
action. Laboratory corrective action occurs at several levels. The most 
common and efficient corrective action involves the action of the 
technician or analyst in charge of analyzing a batch of samples. In most 
analytical procedures, nonconformity may be signaled by significant 
deviations in instrument response, variability in replicate analyses of 
a standard or sample, atypical blank responses, or other unusual 
characteristics. The technician or analyst then may attempt to locate 
the cause of the nonconformity and effect correction prior to 
calibrating and running the samples. Results of QC samples may also 
signal nonconformity and can also trigger corrective action. Although 
variations in accuracy and precision reflected in QC samples are 
typically determined well after a batch of samples has been run, the 
analyst or technician may also note unusual responses for some blanks, 
replicates, or reference samples that may immediately be brought to the 
attention of the Laboratory Director for more immediate corrective 
action. 

\noindent
The following guidelines are used to evaluate nonconformity in trace 
organic QC samples, method or calibration blanks, or surrogates: 

\begin{enumerate}
	\item Upper Control Limits (UCL) exceeded for Range (R) and Recovery (P) 
	\item Lower Control Limits (LCL) exceeded for Recovery (P)
	\item Blanks exceeding Reporting Limits (QL)
	\item Failure of Performance Evaluation (PE) sample analysis
\end{enumerate}

\noindent
 If corrective action is necessary, the analyst and Laboratory 
Director will take some or all of the following steps to remedy the 
problem:

\newcounter{numberedCntB}
\begin{enumerate}
	\item Check methodology to verify preparation and analytical SOPs were followed
\setcounter{numberedCntB}{\theenumi}
\end{enumerate}


\begin{enumerate}
\setcounter{enumi}{\thenumberedCntB}
	\item Check calculations and measurement data
	\item Check instruments to ensure proper calibration and operation
	\item Check reagents and laboratory conditions for contamination
	\item Reanalyze all samples run at the time the problem was detected and compare original to re-run values to verify matrix effects or contamination
	\item If the problem is not resolved, seek assistance from instrument manufacturer
\setcounter{numberedCntB}{\theenumi}
\end{enumerate}

\noindent
The following guidelines are used if the nonconformity is in the 
instrument tune or calibration:

\newcounter{numberedCntD}
\begin{enumerate}
	\item Check the maintenance logs and associated instrumentation and columns. Perform maintenance if required
	\item Check expiration dates and integrity of standards. Re-prepare standards as necessary.
	\item Determine if sample results are affected
	\item Re-calibrate instrument to meet specifications and re-analyze samples
	\setcounter{numberedCntD}{\theenumi}
\end{enumerate}
 

Reports of quality control results are prepared annually, or more 
frequently if problems arise, and submitted to the Laboratory Director. 
These reports consist of a summary of quality control calculations for 
the year and a comparison to the previous year's results. Any changes in 
control limits, analytical variability, or other problems will be noted 
in the report together with recommendations for improvements or 
modifications to the analytical process. 



\section{Data Reduction, Validation and Reporting}
The technician or analyst in charge of the analysis is 
responsible for verifying and tabulating raw data into a form containing 
the Lab ID\#, Field identifier, collection date, project, protocol, 
batch number, analysis date, and results of analysis. The analyst 
reviews the tabulated results to verify that sample preparation/analysis 
documentation is correct and complete, the appropriate SOP was followed, 
QC results are within control limits, and that any special sample 
preparation/analysis requirements have been met. The WSL standard 
operating procedure Gen-Batch Accept and Report-001 lists general 
acceptance and reporting procedures .

Any problems with sample analysis will be communicated verbally and in 
writing to the supervisor, together with an explanation of how the 
problem was resolved. Calculations for data reduction are included in 
the method's standard operating procedure. Results are typically entered 
or transferred electronically to a computer spreadsheet for performing 
calculations and reporting, although handwritten results are acceptable. 
The data package is then initialed, dated, and passed on for review.

Data review and validation may be performed by both a supervising 
chemist and Laboratory Director and includes calculation of quality 
control statistics (range and recovery). Data review includes a check of 
calibration data, QC results, completeness of supporting documentation 
and results, and determination if results are ready for release in the 
form of a final report. If concentrations are not already in standard 
units, results are converted to mg/L or ?g/L for liquid samples, ?g/g or 
ng/g for solid samples, and ?L/L or nL/L for gaseous samples, with 
method sensitivity determining the appropriate range. Results falling 
below the most recent reporting limits are converted to "\,$<$reporting 
limit" unless the project or individual requesting the analyses 
specifies uncensored results. A disclaimer is added to uncensored 
results indicating that concentrations below reporting limits are 
indeterminate and cannot be verified. 

Quality control results falling outside control limits are immediately 
subjected to corrective action as discussed in the previous section. If 
corrective action does not resolve the nonconformity and the source of a 
problem cannot be identified, the results for the affected sample batch 
are reported with a footnote describing the quality control issue. If 
the source of the problem can be identified but cannot be corrected, the 
results may be discarded and the sampler or other responsible party will 
be contacted to determine whether re-sampling or other alternatives can 
be arranged in order to provide valid results. Issues that affect data 
quality are included in the cover letter or narrative that is produced 
with the sample results. 



\section{Documentation and Records}
The most recent versions of the quality assurance manual, 
standard operating procedures, and other relevant documents are 
distributed to affected laboratory staff, and a complete set of 
documents is available at all times in the main sample preparation 
laboratory (Room 203). Only the most current version of any document is 
available to staff in electronic copy and the laboratory WSLims is 
defined to all laboratory personnel as the current reference for each 
document. Older versions of these documents are collected and held by 
the Laboratory Director until they are no longer needed. A revision 
number is indicated in the 3-digit code included in the document or 
method number (see Gen-WSL SOP Format-003 -Format for Standard Operating 
Procedures). The Laboratory Director will be responsible for ensuring 
that the most recent versions of all documents are used by laboratory 
staff, and that the most recent versions of documents are available in 
the WSLims. Other records include, and are not limited to, personnel 
records, QA corrective action files, laboratory notebooks and 
worksheets, bench sheets, maintenance logs, standards logs, and 
laboratory sample log-in files. Sample log-in information is also held 
in the WSLims as noted below. Records are stored in designated file 
drawers or electronically and are retained for 5 years, or as specified 
by contract, to allow for accessing raw data information. 

Holding times are calculated from the collection and preparation dates 
and stored by the WSLims. Samples are typically stored until results are 
verified and reported, and may be held until the results have been 
released and delivered to ensure that reanalysis will not be required. 
Results for samples prepared and analyzed after the maximum holding 
times have expired will be flagged. Results and supporting documentation 
may be held indefinitely at the Water Sciences Laboratory although data 
older than five years may not be verifiable. Raw results are held in 
files, notebooks, and other standard forms. Electronic raw results and 
data are archived on magnetic tape. Electronic records are secured 
through a digital signature. WSL staff are assigned unique names and 
each person chooses an individual password. To log into laboratory 
computers both the unique name and password are required.

 

\section{Laboratory Organization and Responsibility}
The WSL Quality Assurance Program is primarily the 
responsibility of the Laboratory Director. The manager is responsible 
for designing, equipping, and monitoring the laboratory quality 
assurance program including operating procedures, laboratory records, 
statistical techniques, calibration, and equipment maintenance.

The Laboratory Director will manage and provide oversight for the 
Quality Assurance Programs. The Laboratory Director does not perform the 
sample analysis and is independent from data generating groups. All 
corrective action is approved by the Laboratory Director and he/she has 
final authority to stop work or make substantial changes to any method 
or procedure. The Laboratory Director monitors QC activities and 
results, determines conformity of procedures and results, and makes 
appropriate recommendations for corrections and improvements. The 
Laboratory Director seeks out new ideas and current developments in the 
field of quality control and makes recommendations for possible 
improvements where appropriate. The Laboratory Director is responsible 
for periodic review of the quality assurance manual to ensure that it 
reflects the current needs and operating conditions of the WSL. 
Revisions to the quality assurance program may become necessary 
following internal audits, assessments, inspections, or site visits. 

Most laboratory staff hold degrees in environmental sciences, chemistry 
or laboratory technology. The minimum educational level of professional 
level staff is a bachelor's degree with experience, or a master's 
degree. Technical staff may possess a bachelor's degree (Grade III) or 
an Associate's Degree (Grade II). Laboratory technicians are typically 
recruited with Environmental Laboratory degrees as well as experience in 
an analytical laboratory. New personnel are given a concise summary of 
their job responsibilities, trained and tested in specific analytical 
methods and basic quality assurance/control procedures by experienced 
staff members before handling and analyzing samples. 



\section{Procurement}
Purchased equipment, supplies, reagents, standards and other 
testing materials must be of sufficient quality so as not to adversely 
affect analytical results. Scientific vendors are regarded as resources 
or extensions of the analytical laboratory (Ratliff, 1990), and thus 
must adhere to the same standards of quality. The WSL has access to and 
experience with a wide variety of scientific manufacturers, both 
directly and indirectly through the University Purchasing Department. 
The Laboratory also is fortunate in most, if not all, cases to have the 
final word in choosing a supplier. 

The Purchasing Department manages a systematic procurement process 
providing for the cost-effective acquisition of quality goods and 
services in a reasonable time frame. It is responsible for organizing 
and administering a centralized purchasing service for all departments 
in accordance with federal regulations, state laws, Board of Reagents 
bylaws and policies, and UNL purchasing procedures. Orders for equipment 
and supplies are generally placed directly with the appropriate vendor 
after obtaining a purchase order from the University Purchasing 
Department.
